\documentclass[12pt,pdf,hyperref={unicode}]{beamer}
\usepackage[utf8]{inputenc}
\usepackage[russian]{babel}
\usepackage{amsmath, amsthm}
\usepackage[export]{adjustbox}
\usepackage{amsfonts}
\usepackage{pgfplots}
\usepackage{tikz}
\usepackage{adjustbox}
\usepackage{listings}
\usepackage[T2A]{fontenc}
\usepackage[koi8-r]{inputenc}%включаем свою кодировку: koi8-r или utf8 в UNIX, cp1251 в Windows
\renewcommand{\lstlistingname}{Листинг}
\usepackage{python}

\documentclass{article}
\usepackage{listings}
\usepackage{xcolor}

\definecolor{codegreen}{rgb}{0,0.6,0}
\definecolor{codegray}{rgb}{0.5,0.5,0.5}
\definecolor{codepurple}{rgb}{0.58,0,0.82}
\definecolor{backcolour}{rgb}{0.95,0.95,0.92}

\lstdefinestyle{mystyle}{
    backgroundcolor=\color{backcolour},   
    commentstyle=\color{codegreen},
    keywordstyle=\color{magenta},
    numberstyle=\tiny\color{codegray},
    stringstyle=\color{codepurple},
    basicstyle=\ttfamily\footnotesize,
    breakatwhitespace=false,         
    breaklines=true,                 
    captionpos=b,                    
    keepspaces=true,                 
    numbers=left,                    
    numbersep=5pt,                  
    showspaces=false,                
    showstringspaces=false,
    showtabs=false,                  
    tabsize=2
}

\lstset{style=mystyle}

\usepackage{graphicx,xcolor}
\usepackage{algorithm2e}
\usepackage{pdfpages}

\usetheme{Darmstadt}%Darmstadt
\setbeamertemplate{\insertframenumber}

\newtheorem{statement}{Утверждение.}

\setcounter{enumi}{1}
\title{Практикум на ЭВМ задание 4}
\author{Титушин Александр, Тишина Ульяна, Суроева Кристина}
\institute{группа 311}

\begin{document}
\newcommand\Rn{\R^n}

\frame{\maketitle}


\begin{frame}\frametitle{Постановка задачи}
Два основных поставщика: Harpy \& Co и Westeros Inc.
На протяжении нескольких месяцев мы закупаем сталь у обеих
компаний, каждая из которых предлагает ощутимую скидку при
заключении эксклюзивного договора на поставку.
У нас есть записи о производстве мечей, а также данные о
количестве сломанных мечей в каждый из месяцев ведения
боевых действий.
Проведем разведовательный анализ данных и попробуем
выбрать предпочтительного для нас поставщика.
	
\end{frame}

\begin{frame}\frametitle{Общая статистка по мечам}
Для начала узнаем, сколько всего мечей из стали каждого
поставщика было произведено и сломано.
Из диаграммы ниже видно:
\begin{figure}
		\includegraphics[height=0.3\textwidth]{./statistics_1.png}
\end{figure}
• разница в производстве мечей едва заметна у Westeros Inc больше на
$0,29\% $ ); \newline • разница в количестве сломанных мечей существенна: из
стали Harpy \& Co сломанных мечей на $36\%$ меньше.
\end{frame}

\begin{frame}\frametitle{Cтатистика по месяцам}
Посмотрим на количество произведенных и сломанных мечей
из каждой стали в каждый месяц:
\begin{figure}
		\includegraphics[height=0.6\textwidth]{./statistics_2.png}
\end{figure}
\end{frame}

\begin{frame}\frametitle{Cтатистика по месяцам}
Из графика видно: \newline 
\newline
1) Темпы производства мечей из стали обеих компаний
стабильно находятся примерно на одном уровне на
протяжении всех 6 месяцев (на 7-ом месяце
производства не было);
\newline 
\newline
2) Видно, что из стали Harpy \& Co каждый месяц, кроме
7-ого, ломалось значительно меньше мечей, при этом
7-ой месяц не оказал существенного влияния: среднее
значение поломок (они отмечены пунктиром
соответствующего цвета для обеих компаний) все равно
меньше у мечей из стали Harpy \& Co на те же 36\%.
\end{frame}

\begin{frame}\frametitle{Cтатистика по кузнецам}
Мечи из стали каждой
компании производят по 50
кузнецов. На графике
они изображены точками, по
которым видно, что модели
мечей из стали Harpy \& Co
стабильно показывают лучший
результат по кузнецам, ломаясь
в среднем на 36\% реже чем у Westeros Inc при
схожих темпах производства.

\begin{figure}
		\includegraphics[height=0.4\textwidth]{./statistics_3.png}
\end{figure}

\end{frame}

\begin{frame}\frametitle{Выводы}
Из проведенного нами анализа данных можно с большой
уверенностью сделать вывод, что эксклюзивный договор однозначно стоит
заключить с поставщиком Harpy \& Co, так как мечи из их
стали в среднем ломаются реже вне зависимости от даты
производства и кузнеца.
\end{frame}
\end{document}
